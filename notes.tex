\documentclass{article}
\usepackage{graphicx}
\usepackage{subcaption}
\usepackage{listings} % Include the listings-package
\usepackage{xcolor} % Required for custom colors
\usepackage{fullpage}
\usepackage{enumitem}
\usepackage{upquote} % for straight quotes in texttt
\usepackage{siunitx}
\usepackage{hyperref}
\usepackage{amsmath}
\usepackage{amssymb}

\usepackage{tikz}
\usetikzlibrary{shapes.geometric, arrows.meta, positioning}


% If dealing with URLs or file paths:
\usepackage{url} % Allows for line breaks in URLs
% \def\UrlBreaks{\do\/\do-} % Allows breaks at "/" and "-"

\usepackage[style=authoryear-comp,  % numeric, alphabetic, authoryear, etc.
    sorting=nty,
    abbreviate=true,
    backend=biber,
    maxcitenames=2,
    mincitenames=1,
    uniquelist=false]{biblatex}
\addbibresource{biblatex.bib}
\usepackage{pgfplots}
\pgfplotsset{compat=1.16}
\usetikzlibrary{intersections}
\usepgfplotslibrary{fillbetween}
\usepackage{subcaption}

\renewbibmacro*{name:andothers}{% Based on biblatex.def
  \ifboolexpr{
    test {\ifnumequal{\value{listcount}}{\value{liststop}}}
    and
    test \ifmorenames
  }
    {\ifnumgreater{\value{liststop}}{1}
       {\finalandcomma}
       {}%
     \textnormal{\textnormal{ et al.,}}} % Here we ensure "et al." is not bold
    {}}
\renewcommand*{\finalnamedelim}{\addspace\textnormal{and}\addspace}

% Define new commands for \left( and \right)
\let\originalleft\left
\let\originalright\right
\renewcommand{\l}{\originalleft}
\renewcommand{\r}{\originalright}

% To make vectors
\renewcommand{\v}[1]{\vec{#1}}





% To add Python code
\usepackage{tcolorbox}
\tcbuselibrary{minted,breakable,xparse,skins}
\usepackage{caption} % Add this line
\definecolor{bg}{gray}{0.95}
% Modified mintedbox
\DeclareTCBListing{mintedbox}{O{}m!O{}}{
  breakable=false, % Set to false to prevent page breaks
  listing engine=minted,
  listing only,
  minted language=#2,
  minted style=default,
  minted options={
    %linenos, % This option enables line numbers
    gobble=0,
    breaklines=true,
    breakafter=,
    fontsize=\small,
    numbersep=8pt,
    #1},
  boxsep=0pt,
  left skip=0pt,
  right skip=0pt,
  left=25pt,
  right=0pt,
  top=3pt,
  bottom=3pt,
  arc=5pt,
  leftrule=0pt,
  rightrule=0pt,
  bottomrule=2pt,
  toprule=2pt,
  colback=bg,
  colframe=orange!70,
  enhanced,
  overlay={
    \begin{tcbclipinterior}
    \fill[orange!20!white] (frame.south west) rectangle ([xshift=20pt]frame.north west);
    \end{tcbclipinterior}},
}



\hypersetup{
    colorlinks=true,
    linkcolor=cyan,     % Color for internal links
    citecolor=cyan,     % Color for citations
    filecolor=cyan,     % Color for file links
    urlcolor=cyan       % Color for URLs
    % pdfborder={0 0 0},
}

% Define the amber color
\definecolor{amber}{HTML}{FFBF00}% Define the specific light shade of color
\definecolor{light083676}{HTML}{083676}
\definecolor{action}{HTML}{7A001E}

% Define a new command for small caps text with the defined color
\newcommand{\textq}[1]{{\color{light083676}\textsc{#1}}}
\newcommand{\key}[1]{\fbox{#1}} % Create a new command to format keys as buttons
\let\oldtexttt\texttt
\renewcommand{\texttt}[1]{\textcolor{magenta}{\oldtexttt{#1}}}
% Define text for warning (where something might cause the research to fail)
\newcommand{\textw}[1]{{\color{amber!70}#1}}
\newcommand{\texta}[1]{{\color{action}#1}}

\newcommand\tab[1][1cm]{\hspace*{#1}}
\makeatletter
\newcommand\footnoteref[1]{\protected@xdef\@thefnmark{\ref{#1}}\@footnotemark}
\makeatother

\lstset{
    basicstyle=\ttfamily,
    commentstyle=\color{gray},
    keywordstyle=\color{blue},
    breaklines=true,
    numbers=left,
    numberstyle=\tiny\color{gray},
    stepnumber=1,
    frame=single,
    language=Python,
    showstringspaces=false,
}

\title{Spin Precession and Rate Equations}
\author{Noor Almusleh}
\date{\today}

\begin{document}

\maketitle


\textbf{Spin precession} is related to the time evolution of two spin components, where each has a different time-dependent phase term, i.e., measuring a spin component change with time. This can be described, like any other transition in a physical system, by the \textbf{rate equation}. Spin plays an important role in the study of particles and in quantum information. For example, we can measure a quantum system by interacting it with a light beam. Light itself is comprised of electric and magnetic fields that are perpendicular to each other. These two fields can be described by the following two equations:
\begin{align}
    \label{eq:E}
    \Vec{E} (\Vec{r}, t) &= E_0(\omega) \Vec{\varepsilon}\sin(\Vec{k}\cdot\Vec{r} - \omega t + \alpha) \\
    \label{eq:B}
    \Vec{B} (\Vec{r}, t) &= E_0(\omega) \frac{1}{\omega} (\Vec{k} \times \Vec{\varepsilon}) \sin(\Vec{k}\cdot\Vec{r} - \omega t + \alpha)
\end{align}

where $\Vec{E}$, and $\Vec{B}$ are the electric and magnetic fields respectively, $\Vec{k}$ is the wavenumber, and $\Vec{\varepsilon}$ is the polarization vector, which has a \textw{direction that is perpendicular to the electric and magnetic fields, i.e. is in the propagation direction}. The superposition of frequency can be used to give \textw{an indication of the polarization.} Additionally, the speed of light in vacuum can be given in terms of the dielectric permittivity in free space, and magnetic permeability in free space, that is
\begin{equation}
    c^2 = \frac{1}{\epsilon_0 \mu_0}
\end{equation}

The energy density of an electromagnetic wave is given by 
\begin{equation}
    \label{eq:energy_density_em}
    U(\omega) = \frac{1}{2} \l( \epsilon_0 |\Vec{E}|^2 + \frac{1}{\mu_0} |\Vec{B}|^2 \r)
\end{equation}
therefore, we can substitute eqs. \ref{eq:E} and \ref{eq:B} as follows (n.b. $\eta=\Vec{k}\cdot\Vec{r} - \omega t + \alpha$):
\begin{align*}
    U(\omega) &= \frac{1}{2} \l( \epsilon_0 \l| E_0(\omega) \Vec{\varepsilon}\sin(\eta) \r|^2  + \frac{1}{\mu_0} \l| E_0(\omega) \frac{1}{\omega} (\v{k} \times \v{\varepsilon}) \sin(\eta) \r|^2\r) \\ 
    &= \frac{1}{2} \l( \epsilon_0 E_0^2|\varepsilon|^2 \sin^2(\eta) + \frac{1}{\mu_0\omega^2}E_0^2 |(\v{k} \times \v{\varepsilon})|^2 \sin^2(\eta)  \r) \\
    &= \frac{1}{2} \l(  \epsilon_0 E_0^2|\varepsilon|^2 \sin^2(\eta) + \frac{1}{\mu_0\omega^2}E_0^2 \l| |\v{k}||\hat{\varepsilon}| \sin \frac{\pi}{2} \r|^2 \sin^2(\eta)  \r) \\
    &= \frac{1}{2} \l(  \epsilon_0 E_0^2|\varepsilon|^2 \sin^2(\eta) + \frac{1}{\mu_0\omega^2}E_0^2 \l| \v{k} \r|^2 \sin^2(\eta)  \r) \\
    &= \frac{1}{2} \l(  \epsilon_0 E_0^2|\varepsilon|^2 \sin^2(\eta) + \frac{1}{\mu_0\omega^2}E_0^2 \l| \frac{2\pi}{\lambda} \r|^2 \sin^2(\eta)  \r) \\
    &= \frac{1}{2} \l(  \epsilon_0 E_0^2|\varepsilon|^2 \sin^2(\eta) + \frac{\epsilon_0}{
c^2\omega^2}E_0^2 \l| \frac{2\pi \nu}{c} \r|^2 \sin^2(\eta)  \r) \\
    &= \frac{1}{2} \l(  \epsilon_0 E_0^2|\varepsilon|^2 \sin^2(\eta) + \epsilon_0E_0^2 \sin^2(\eta)  \r) \\
    &=  \epsilon_0 |E_0|^2|\varepsilon|^2 \sin^2(\eta)  \\
    \blacksquare
\end{align*}

It is also worth mentioning that $E_0$ could depend on $\omega_0$ \textq{what does that mean?} If we average the energy density over a period of time. This will be essential when looking for a linking the particle-picture and the wave-picture. The \textbf{time-averaged energy density} is given by:
\begin{equation}
    \label{eq:time_averaged_energy_density}
    \rho (\omega) = \frac{1}{T} \int_0^T dt U(\omega) = \frac{1}{2}\epsilon_0 |E_0(\omega)|^2
\end{equation}

Given that the number of photons, with angular frequency $\omega$, within a volume $V$ is $N(\omega)$, and and their time-averaged energy density $\rho(\omega) = \hbar \omega N(\omega) / V$, using eq. \ref{eq:time_averaged_energy_density} we can deduce a relationship between the electric field and the number of photons, that is,
\begin{equation*}
    \frac{\hbar \omega N(\omega)}{V}    =   \frac{1}{2}\epsilon_0 |E_0(\omega)|^2
\end{equation*}
\begin{equation}
    \label{eq:enectric_field_num_photons}
    E_0(\omega) = \sqrt{2 \frac{\hbar \omega}{\epsilon_0}\frac{N(\omega)}{V}}
\end{equation}












\end{document}
